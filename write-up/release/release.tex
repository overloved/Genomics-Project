\documentclass[a4paper,12pt]{article}
%\documentclass[a4paper,12pt]{amsart}
\author{Xiang Niu, Yao Huang}
\title{The \TeX\ Book}
\date{\today}
\usepackage{amsmath}
\usepackage{amsthm}
\begin{document}
\maketitle
\section{Abstract}
\section{Introduction}
\section{Prior work}
In ``linear Work Suffix Array Construction''\cite{simple}, an algorithm \textit{DC} is used to construct suffix arrays with a simple linear-time. Give $v \in [1,\sqrt{n}]$, it runs in $O(vn)$ time using $O(n/\sqrt{v})$ space.

\section{Methods and software}
\section{Results}
\section{Conclusion}
\section{Special characters}
\textbf{Exercise 6.} \textit{Prove Fermat's Theorem.}
See table~\ref{tab:test} on page~\pageref{tab:test}.
\subsection{Accents}
\begin{tabular}{l r r r c} 
Name&Exam1&Exam2&Exam3&Grade\\
John&19& 28&33&C \\
Jane&49& 35&60&B  \\
Jim&76& 38&59&A 
\end{tabular}

\begin{table}
\begin{center}
\begin{tabular}{| l || r | r | r | c |}  
\hline
Name&Exam1&Exam2&Exam3&Grade\\
\hline\hline
John&19& 28&33&C \\  
\hline
Jane&49& 35&60&B  \\
\hline
Jim&76& 38&59&A  \\
\hline
\end{tabular}
\caption{Math 361 Grades\label{tab:test}} 
\end{center}
\end{table}

\subsection{Braces}
$\mathbf{R}$

\smallskip
\noindent\textbf{Fermat's Last Theorem.}
\textit{For $n\ge3$, 
the equation $x^n+y^n=z^n$ has no non-trivial integer solutions.}
\smallskip
\begin{proof} See Wiles.
\end{proof}

\begin{itshape}
The polynomial $p(t)$ splits...
\end{itshape}

\subsection{Dollar signs}
\[
\left|\sum_{i=1}^n (\quad's) a_ib_i\right|
\le
\left(\sum_{i=1}^n a_i^2\right)^{1/2}
\left(\sum_{i=1}^n b_i^2\right)^{1/2}
\]

\section{Sectioning}
\begin{center}123 \end{center}
\begin{quote} 345\end{quote}

\begin{itemize}
\item This is the first item
\item This is the second item
\item This is the last itme
\end{itemize}

\smallskip
\textbf{Some special characters in TeX:}\cite{simple,parallel}
\begin{enumerate}
\item Accents  
\item Braces 
\item Dollar signs 
\end{enumerate}

\begin{equation}
\sum_{k=1}^n k^2, \frac{a}{q}, \int_1^x\frac{1}{x}dx, \sin(x), \arcsin(x), e^{2 \pi i}
\end{equation}
\begin{align*} 
(a+b)^3 &= (a+b)^2(a+b)\\
&=(a^2+2ab+b^2)(a+b)\\
&=(a^3+2a^2b+ab^2) + (a^2b+2ab^2+b^3)\\
&=a^3+3a^2b+3ab^2+b^3
\end{align*}
\section{Conclusion}
1d23

\begin{thebibliography}{99}
\bibitem{simple}
Juha Karkkainen and Peter Sanders,
\emph{Simple linear work suffix array construction.}
Automata, Languages and Programming, pages 187–187, 2003.
\bibitem{parallel}
Fabian Kulla and Peter Sanders,
\emph{Scalable parallel suffix array construction.}
Parallel Computing, 33(9):605–612, 2007.
\end{thebibliography}

\end{document}
